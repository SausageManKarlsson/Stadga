\documentclass[11pt, includeaddress]{classes/cthit}
\usepackage{titlesec}
\usepackage{verbatimbox}

\titleformat{\paragraph}[hang]{\normalfont\normalsize\bfseries}{\theparagraph}{1em}{}
\titlespacing*{\paragraph}{0pt}{3.25ex plus 1ex minus 0.2ex}{0.7em}

\graphicspath{ {images/} }

\begin{document}

\title{Verksamhetsplan \STYRITFULL{}'14/15}
\approved{2014--10--09}
\maketitle

\thispagestyle{empty}

\newpage

\makeheadfoot%

%Rubriksnivådjup
\setcounter{tocdepth}{2}
%Sidnumreringsstart
\setcounter{page}{1}
\tableofcontents

\newpage

\section{Förord}
Verksamhetsplanen för styrIT 14/15 har tagit sin form ur den preliminära verksamhetsplanen som styrIT 13/14 arbetade fram tillsammans med aspar. Planen grundar sig i sektionens mål- och visionsdokumentet, det kontinuerliga arbetet och de fokusområden styrIT väljer. Verksamhetsplanen kommer totalt sett täcka samtliga mål ur mål- och visionsdokumentet.


\section{Operativt arbete}
De kontinuerliga åtaganden styrIT har kommer under året uppta den största delen av arbetet. Detta jobb består av att:
\begin{itemize}
	\item Gå på de olika utskott, arbetsgrupper samt övriga möten som styrIT skall vara representerade på
	\item Behandla frågor rörande sektionen som helhet
	\item Behandla incidenthanteringar
	\item Utbilda samt hjälpa sektionsaktiva
	\item Behandla inkomna äskningar
	\item Vara delarrangör av Kandidatmiddagen
	\item Arrangera sektionsmöten
	\item Underhålla styrdokument
\end{itemize}

\section{Fokusområden}
Utöver det kontinuerliga arbetet kommer styrIT14/15 jobba med några olika fokusområden för att belysa extra viktiga frågor som måste fungera bra för att sektionen ska må och fungera väl eller för att utveckla delar av sektionen som inte håller tillräckligt hög nivå.

\subsection{Jämlikhet}
På vår sektion finns det många stora problem inom jämlikhet vilket är något som vi måste jobba ännu bättre för. Det är främst två klyftor mellan olika grupper samt en mer generell fråga som arbetet kommer gå ut på. Den ena klyftan är mellan de som kan programmera när de kommer till IT-sektionen och de som inte kan. Arbetet med den frågan är något som kommer bedrivas tillsammans med programledningen och snIT främst. Den andra klyftan är den mellan antalet kvinnor och män på sektionen vilket är ett otroligt stort och svårt problem och därför är det viktigt att sektionen jobbar med frågan på ett långsiktigt sätt. För att arbetet med dessa två klyftor ska vara effektivt och framgångsrikt är det också viktigt att jobba med den mer generella frågan kring attityder på sektionen hos alla medlemmar.
\newline
\newline
Detta område och dessa punkter riktar in sig mot många av de mål som finns i mål- och visionsdokumentet men främst mot: 5, 6, 7, 9 och 10.

\subsection{Ekologisk hållbarhet}
Våren 2013 klubbades en miljöpolicy genom på sektionen, men detta har varken publicerats eller uppmärksammats. Vi tror dessutom att instruktionerna i denna policy i sin form inte uppmuntrar kommittéer att välja ekologiska och etiska produkter när de ställs emot billigare produkter. styrIT kommer därför att arbeta aktivt för ett generellt miljötänk på sektionen och på så sätt bidra till att policyn efterlevs. Till att börja med kommer vi välja ekologiskt i de fall där styrIT själva har möjlighet att påverka diverse inköp.
\newline
\newline
Detta område och dessa punkter riktar främst in sig mot mål 11 som finns i mål- och visionsdokumentet.

\subsection{Kommunikation}
Under det gångna året har styrIT arbetat med frågan kring kommunikation och försökt synliggöra det arbete sektionen och styrIT gör. Detta är något som styrIT14/15 kommer fortsätta arbeta med då det är en viktig fråga som det finns mycket kvar att utveckla inom. Arbetet med frågan kommer också handla om att kommunicera befintlig information på ett bättre sätt, vilket bland annat innebär att översätta relevanta dokument och hemsida till engelska då många medlemmar på sektionen inte förstår svenska. Även kommunikation åt andra hållet, in mot styrIT, kommer arbetas med för att medlemmar lättare ska kunna påverka och ge förslag till saker som sektionen ska arbeta med.
\newline
\newline
Detta område och dessa punkter riktar in sig mot många av de mål som finns i mål- och visionsdokumentet men främst mot: 11, 13, 15 och 16.

\subsection{En enad sektion}
En av de viktigaste delarna i styrITs arbete är att utveckla relationen mellan sektionens alla organ (FKIT). Detta är något som arbetades hårt med under 13/14 och är något som är viktigt att fortsätta med. Utöver det kontinuerliga arbetet med att hjälpa FKIT när det behövs så kommer arbetet även gå ut på att arrangera utbildningar, workshops och teambuildingar för FKIT. En del av arbetet kommer även gå ut på att främja samarrangemang mellan olika delar av FKIT samt fortsätta arrangera kvällar inriktade till specifika poster i de olika delarna av FKIT.
\newline
\newline
Detta område och dessa punkter riktar främst in sig mot mål 7 och 8 i mål- och visionsdokumentet, men är otroligt viktigt och grunden till att stora delar av sektionens verksamhet skall fungera på ett bra sätt.

\subsection{Master}
Det område IT-sektionen ligger längst efter i är inom våra masterprogram. En anledning till det är självklart att de inte befinner sig på samma geografiska plats som kandidatdelen, men arbetet med att inkludera de medlemmar som läser masterstudier har varit bristfälligt. För att vår sektion ska vara så inkluderande som vi vill är det otroligt viktigt att vi därför börjar arbeta med denna fråga. Precis som att Mottagningen är viktig för Nollan så tror styrIT att det är viktigt med en slags mindre mottagning för mastereleverna vilket målet är att arrangera. Därefter är det viktigt att hålla kontinuerlig kontakt genom att arrangera passande arrangemang för studenterna på Lindholmen.
\newline
\newline
Detta område och dessa punkter riktar in sig mot många av de mål som finns i mål- och visionsdokumentet men främst mot: 6, 7, 9, 15 och 16.

\subsection{Lokaler}
IT-sektionen har under de senaste åren växt och det ser inte ut som att sektionen kommer minska i storlek igen på många år. I dagsläget kan sektionslokalen endast rymma 75 personer samtidigt vilket är alldeles för lite för en sektion som vår där sammanhållningen är otroligt god och där intresset för att vara i sektionslokalen är högt. Därför är det viktigt att sektionen hela tiden fortsätter jobba för bättre och större lokaler samtidigt som man hela tiden jobbar för att utnyttja de lokaler som finns att tillgå på bästa sätt.
\newline
\newline
Detta område och dessa punkter riktar in sig mot många av de mål som finns i mål- och visionsdokumentet men främst mot: 5, 6 och 10.

\section{Övrigt}
Det finns många tankar och idéer som styrIT vill genomföra som är viktiga men som inte på ett naturligt sätt passar in under ett fokusområde. Några av dessa är att föra över styrITs egna sida styrit.chalmers.it till sektionens hemsida www.chalmers.it för att underlätta för medlemmar att hitta information om sektionsstyrelsen och att utvärdera möjligheten och om fördelaktigt skapa en medlemsinitiativsfond som sektionsmedlemmar kan äska pengar ur för att genomföra projekt som gynnar medlemmarna på sektionen. Styrdokumenten kommer dessutom att revideras grundligt för att bli tydliga och konsekventa.

\end{document}

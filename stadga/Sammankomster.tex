\section{Sammankomster}

\subsection{Årsmöte}

\subsubsection{Kallelse}
Föreningens styrelse beslutar om tid och plats för föreningens ordinarie årsmöte.

\subsubsection{Mötets behörighet} \label{sec:mötets-behörighet}
För att vara behörigt måste årsmötet utlysas minst tio läsdagar i förväg och referens till dagordningen finnas tillgänglig via sektionens anslagstavla, sektionens hemsida samt föreningens maillista. Kallelsen till ordinarie årsmöte skall även innehålla hänvisningar till ekonomisk berättelse och budget.

\subsubsection{Dagordning}
Följande punkter måste behandlas på det ordinarie mötet:
\begin{itemize}
    \item Mötets öppnande
    \item Val av mötets ordförande
    \item Val av mötets sekreterare
    \item Val av mötets två rösträknare tillika justerare
    \item Fastställande av röstlängd
    \item Mötets behörighet
    \item Fastställande av dagordningen
    \item Verksamhetsberättelse
    \item Ekonomisk berättelse
    \item Revisionsberättelse
    \item Ansvarsfrihet för förra årets styrelse
    \item Verksamhetsplan
    \item Budget
    \item Motioner
    \item Val av styrelse
    \item Val av revisorer
    \item Övriga frågor
    \item Mötets avslutande
\end{itemize}

Följande punkter måste behandlas på extra årsmöte:
\begin{itemize}
    \item Mötets öppnande
    \item Val av mötets ordförande
    \item Val av mötets sekreterare
    \item Val av mötets två rösträknare tillika justerare
    \item Fastställande av röstlängd
    \item Mötets behörighet
    \item Fastställande av dagordningen
    \item Övriga frågor
    \item Mötets avslutande
\end{itemize}

\subsubsection{Extra årsmöte}
Om styrelsen eller minst hälften av föreningens medlemmar vill, eller revisorn kräver det, så skall styrelsen kalla till ett extra årsmöte. Vid giltigt krav på extra årsmöte kan den som krävt det sköta kallelsen. För att vara behörigt gäller även här §\ref{sec:mötets-behörighet}. På extra årsmöte kan bara de ärenden som nämnts i kallelsen behandlas.

\subsection{Beslut}

\subsubsection{Röstetal}
Beslut genom röstning sker genom enkel majoritet om inte annat anges i stadgarna. Nedlagda röster räknas ej. Varje person med rösträtt har en röst. Vid lika röstetal får föreningsordförande avgöra, förutom vid personval, i vilket fall lotten får avgöra.

\subsubsection{Stadgeändring} \label{sec:stadgeändringar}
Dessa stadgar kan bara ändras på ordinarie eller extra årsmöte. För att vara giltig måste ändringen antas med två tredjedelar av antalet röster. Då stadgeändring skall ske måste förslaget delges medlemmarna i kallelsen till mötet. I annat fall måste ändringen antas enhälligt. Ändring av föreningens stadgar om föreningsform (§\ref{sec:föreningsform}), syfte (§\ref{sec:syfte}), stadgeändringar (§\ref{sec:stadgeändringar}) och upplösning (§\ref{sec:upplösing}) kräver beslut på ordinarie årsmöte.

\subsubsection{Upplösning av föreningen} \label{sec:upplösing}
Förslag om föreningens upplösning får endast framläggas på årsmöte. Att upplösning skall behandlas måste framgå av kallelsen. Föreningen kan inte upplösas så länge minst fem medlemmar vägrar godkänna upplösningen. Vid upplösning skall föreningens skulder betalas samt all lånad inventarie återlämnas till dess ägare. Därefter skall föreningens tillgångar gå till verksamhet i enlighet med föreningens syfte, om möjligt till liknande förening på sektionen. Hur detta skall ske beslutas på det sista årsmötet.


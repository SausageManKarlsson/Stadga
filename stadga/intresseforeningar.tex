\section{Intresseföreningar}

\subsection{Definition}

\subsubsection{Sammansättning}
Intresseförening är en sammanslutning medlemmar med ett gemensamt intresse. Intresseföreningen ska ha en styrelse bestående av föreningens medlemmar. Minst hälften av medlemmarna i intresseföreningens styrelse ska vara sektionsmedlemmar.

\subsubsection{Medlemsrätt}
Varje sektionsmedlem skall ha rätt till medlemskap. Föreningsmedlem som motverkar föreningens syften kan dock uteslutas av styrelsen. Styrelsen kan även besluta om medlemskap för de som inte är medlemmar av sektionen.

\subsubsection{Stadga}
Intresseförening skall ha en av sektionsstyrelsen godkänd stadga. Dessutom är föreningen skyldig att rapportera till sektionsstyelsen då deras stadga förändrats.

\subsubsection{Syfte}
Intresseförening skall verka för teknologsektionens bästa och ha ett syfte i sin egen stadga.

\subsection{Förteckning}
Teknologsektionens intresseföreningar är de i reglemente förtecknade.

\subsection{Rättigheter}
Intresseförening äger rätt att i namn och emblem använda teknologsektionens namn och symboler.

\subsection{Skyldigheter}
Intresseförening är skyldig att känna till och rätta sig efter teknologsektionens stadgar, reglemente, policies, övriga handlingar och beslut.

\subsection{Ekonomi}
Intresseförening skall ha en från teknologsektionen fristående ekonomi.

\subsubsection{Verksamhet och revision}
Teknologsektionens revisorer har rätt att granska föreningens verksamhet och ekonomi.
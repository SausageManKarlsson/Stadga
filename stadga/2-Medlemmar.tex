\section{Medlemmar}

\subsection{Styrelse} \label{sec:sammansättning-styrelse}

\subsubsection{Sammansättning} 
Föreningens styrelse består av:
\begin{itemize}
    \item Ordförande
    \item Kassör
    \item Samt minst en ledamot
\end{itemize}

\subsubsection{Ansvar}
Styrelsen ansvarar för föreningens medlemslista, medlemsvärvning, sammankomster, beslut som tas på årsmötet och övrig verksamhet.

\subsubsection{Tillsätting}
Styrelsen tillsätts på årsmötet och tillträder sju läsdagar efter valet. Valbar är medlem i föreningen. En medlem kan inte inneha mer än en post i styrelsen.

\subsubsection{Firmateckning}
Rätten att skriva avtal i föreningens namn, föreningens firma, tecknas av ordförande och kassör var för sig. Om särskilda skäl föreligger kan styrelseledamot utses att teckna föreningens firma.




\subsection{Revisorer}
För granskning av föreningens räkenskaper och förvaltning väljs på föreningens årsmöte en eller två lekmannarevisorer. Valbar är person som inte sitter i styrelsen. Revisorerna behöver inte vara medlem i föreningen.



\subsection{Medlemskap}

\subsubsection{Medlemskap}
Som medlem antas intresserad som godkänner dessa stadgar och aktivt tar ställning för ett medlemskap genom att årligen betala föreningens medlemsavgift eller, om medlemskapet är gratis, årligen göra en skriftlig anmälan till föreningen. Avgiftens storlek beslutas på årsmötet. Årsmötet kan besluta att det är gratis att vara medlem.

\subsubsection{Uteslutande}
Styrelsen äger rätt att stänga av en medlem från föreningsaktiviteter om det anses att medlemmen motverkar föreningens syfte.

Sådan avstängning varar i maximalt 42 dagar. Om styrelsen önskar utesluta medlemmen skall ett årsmöte hållas innan denna perioden är över.

Sådant möte skall utlysas stadgeenligt och det skall klart framgå att beslut om medlemmens uteslutande skall fattas på mötet. Beslut om uteslutande fattas med två tredjedelars majoritet, där den berörde medlemmen har rösträtt. Sådant beslut kan inte hävas annat än genom att frågan lyfts på framtida årsmöte. Fram till medlemmens uteslutande har denna fortfarande närvaro-, yttrande- och förslagsrätt på föreningens årsmöten.




\subsection{Medlemsrättigheter}

\subsubsection{Närvaro-, yttrande- och förslagsrätt}
Varje medlem av föreningen har närvarorätt, yttranderätt och förslagsrätt på föreningens samtliga årsmöten. Av årsmötet kan även ickemedlemmar adjungeras så att dessa personer får närvarorätt, yttranderätt och eventuellt förslagsrätt.

\subsubsection{Rösträtt}
Endast fullt betalande närvarande medlem äger rösträtt på årsmöte. På styrelsemöten äger endast närvarande ur styrelsen rösträtt. Röster får alltså varken på styrelsemöte eller årsmöte läggas per proxy.
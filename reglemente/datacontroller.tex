\section{Datahantering}

\subsection{Definitioner}
	    
	    Med dataägare anses den person vars personuppgifter hanteras av organisationen.
	    
	    Hantering av data inkluderar - men är ej begränsat till - lagring, beräkning, använding, visning, distrubution och insamling av data.

\subsection{Personuppgiftsombud}

\subsubsection{Uppdrag}
 Personuppgiftsombudet skall fungera som dataägarnas representant inom organisationen och i sin roll kontrollera att rådande bestämmelser samt riktlinjer kring personuppgifter efterföljs.
 
 Personuppgiftsombudet skall även agera som en kontaktperson för dataägarna gentemot sektionen.

\subsubsection{Sammansättning}
1-4 personer kan utses till personuppgiftsombud för sektionen av sektionsmötet.

Personuppgiftsombud utses för en tid på ett år.

Det är nödvändigt att personuppgiftsombudet kan anses som oberoende och inte har intressen som befinner sig i konflikt till personuppgiftsombudets åligganden.

\subsubsection{Inval}
Personuppgiftsombudet skall varje år väljas på det första ordinarie höstmötet.

\subsubsection{Rättigheter}
Personuppgiftsombudet äger rätt att i namn och emblem använda sektionens namn
och dess symboler.

Personuppgiftsombudet har rätt till utbildning gällande rådande bestämmelser och
riktlinjer för hantering av persondata.

\subsubsection{Åligganden}
\begin{att}
    \item  Granska hur persondata hanteras av organisationen.
    \item  Rapportera brister i hantering av persondata till sektionsmötet och sektionsstyrelsen.
    \item  Känna till vilken hantering av persondata som genomförs av sektionen.
    \item  Rådge organisationen om hantering av persondata.
    \item  Utbilda organisationen om hur de kan följa rådande bestämmelser och riktlinjer för hantering av persondata.
    \item  Inom organisationen representera dataägarnas intressen.
    \item  Svara på frågor från personuppgiftsägare angående organisationens personuppgiftshantering.
    \item  Svara på begäranden från personuppgiftsägare och kontrollera att dessa begäranden
behandlas enligt rådande riktlinjer samt bestämmelser innom organisationen.
\end{att}

\subsubsection{Befogenhet}
Det bifogas personuppgiftsombudet:
\begin{att}
    \item  Vid felaktig hantering, med omedelbar verkan, stoppa hantering av personuppgifter.
\end{att}